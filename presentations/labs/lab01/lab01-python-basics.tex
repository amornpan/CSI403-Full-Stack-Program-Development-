% ============================================
% Lab 1: Python Basics
% CSI403 Full Stack Development
% Date: January 7-9, 2026
% ============================================

% ============================================
% CSI403 Full Stack Development
% Common Preamble for All Presentations
% SPU Chonburi - Semester 2/2568
% ============================================

\documentclass[aspectratio=169,11pt]{beamer}

% ============================================
% PACKAGES
% ============================================
% Theme
\usepackage{../common/beamerthemeSPU}

% Graphics
\usepackage{graphicx}
\graphicspath{{./images/}{../common/images/}}

% TikZ for diagrams
\usepackage{tikz}
\usetikzlibrary{shapes,arrows,positioning,fit,calc,shadows,decorations.pathreplacing}

% Tables
\usepackage{booktabs}
\usepackage{tabularx}
\usepackage{multirow}
\usepackage{colortbl}

% Code listings
\usepackage{listings}
\usepackage{minted}
\usemintedstyle{friendly}

% Boxes
\usepackage{tcolorbox}
\tcbuselibrary{skins,breakable}

% Math
\usepackage{amsmath}
\usepackage{amssymb}

% Icons
\usepackage{fontawesome5}

% Hyperlinks
\usepackage{hyperref}
\hypersetup{
    colorlinks=true,
    linkcolor=spumaroon,
    urlcolor=codeblue,
    citecolor=spumaroon
}

% ============================================
% LISTINGS SETTINGS
% ============================================
\lstset{
    basicstyle=\ttfamily\small,
    backgroundcolor=\color{codebackground},
    keywordstyle=\color{codeblue}\bfseries,
    stringstyle=\color{codegreen},
    commentstyle=\color{codegray}\itshape,
    numberstyle=\tiny\color{codegray},
    numbers=left,
    numbersep=5pt,
    breaklines=true,
    breakatwhitespace=true,
    tabsize=4,
    showspaces=false,
    showstringspaces=false,
    frame=single,
    framerule=0.5pt,
    rulecolor=\color{spulightmaroon},
    xleftmargin=15pt,
    framexleftmargin=15pt,
    aboveskip=10pt,
    belowskip=10pt
}

% Python style
\lstdefinestyle{python}{
    language=Python,
    morekeywords={self,True,False,None,as,with,yield,async,await},
    morestring=[b]""",
    morestring=[b]'''
}

% HTML style
\lstdefinestyle{html}{
    language=HTML,
    morekeywords={DOCTYPE,html,head,body,div,span,class,id,href,src}
}

% SQL style
\lstdefinestyle{sql}{
    language=SQL,
    morekeywords={INT,VARCHAR,DECIMAL,DATETIME,PRIMARY,KEY,FOREIGN,REFERENCES,
                  CREATE,TABLE,INSERT,INTO,VALUES,SELECT,FROM,WHERE,JOIN,ON,
                  UPDATE,SET,DELETE,DROP,ALTER,ADD,INDEX,CONSTRAINT}
}

% Bash style
\lstdefinestyle{bash}{
    language=bash,
    morekeywords={docker,docker-compose,pip,python,pytest,git,curl,cd,ls,mkdir}
}

% ============================================
% TCOLORBOX STYLES
% ============================================
\newtcolorbox{codebox}[1][]{
    colback=codebackground,
    colframe=spumaroon,
    boxrule=1pt,
    arc=3pt,
    fonttitle=\bfseries,
    title=#1
}

\newtcolorbox{infobox}[1][]{
    colback=spulightgray,
    colframe=codeblue,
    boxrule=1pt,
    arc=3pt,
    fonttitle=\bfseries,
    title=#1
}

\newtcolorbox{warningbox}[1][]{
    colback=orange!10,
    colframe=codeorange,
    boxrule=1pt,
    arc=3pt,
    fonttitle=\bfseries,
    title=#1
}

\newtcolorbox{successbox}[1][]{
    colback=green!10,
    colframe=codegreen,
    boxrule=1pt,
    arc=3pt,
    fonttitle=\bfseries,
    title=#1
}

% ============================================
% CUSTOM COMMANDS
% ============================================
% File path
\newcommand{\filepath}[1]{\texttt{\color{codepurple}#1}}

% Terminal command
\newcommand{\cmd}[1]{\texttt{\color{codegreen}\$ #1}}

% URL display
\newcommand{\urlbox}[1]{\fcolorbox{codeblue}{codebackground}{\texttt{\color{codeblue}#1}}}

% Keyboard key
\newcommand{\key}[1]{\fbox{\small\texttt{#1}}}

% Tech logo placeholder
\newcommand{\techlogo}[2][1cm]{\includegraphics[height=#1]{tech-icons/#2}}

% Week indicator
\newcommand{\weekheader}[2]{%
    \begin{tikzpicture}[remember picture,overlay]
        \node[anchor=north east,xshift=-0.5cm,yshift=-0.3cm,
              fill=spugold,text=spumaroon,font=\small\bfseries,
              rounded corners=3pt,inner sep=5pt] 
              at (current page.north east) {Week #1 | #2};
    \end{tikzpicture}
}

% Progress bar
\newcommand{\progressbar}[2]{%
    \begin{tikzpicture}
        \fill[spulightgray] (0,0) rectangle (10,0.3);
        \fill[spumaroon] (0,0) rectangle (#1,0.3);
        \node[anchor=west] at (10.2,0.15) {\small #2};
    \end{tikzpicture}
}

% ============================================
% TITLE PAGE LOGO
% ============================================
\titlegraphic{%
    \includegraphics[height=1.5cm]{spu_logo.png}
}

% ============================================
% COURSE INFORMATION
% ============================================
\newcommand{\coursetitle}{CSI403: Full Stack Development}
\newcommand{\coursesemester}{Semester 2/2568}
\newcommand{\instructor}{Aj. Methas Khamjad}
\newcommand{\university}{Sripatum University Chonburi}

% Default author/institute
\author[\instructor]{\instructor}
\institute[\university]{%
    School of Information Technology\\
    \university
}

\endinput


\title[Lab 1: Python Basics]{Lab 1: Python Basics for Full Stack}
\subtitle{Hands-on Programming Session}
\date{January 7-9, 2026}

\begin{document}

% ============================================
% TITLE SLIDE
% ============================================
\begin{frame}[plain]
    \titlepage
\end{frame}

% ============================================
% OUTLINE
% ============================================
\begin{frame}{Lab Objectives}
    \weekheader{1}{Lab}
    
    By the end of this lab, you will be able to:
    
    \begin{enumerate}
        \item Set up Python development environment
        \item Write and run Python scripts
        \item Understand variables and data types
        \item Use basic input/output operations
        \item Work with strings and numbers
    \end{enumerate}
    
    \vspace{0.5cm}
    
    \begin{alertblock}{Duration}
        2 hours | Individual work | Submit via GitHub
    \end{alertblock}
\end{frame}

% ============================================
% SECTION 1: ENVIRONMENT SETUP
% ============================================
\section{Environment Setup}

\begin{frame}{Step 1: Verify Python Installation}
    \begin{codebox}[Open Terminal / PowerShell]
        \begin{lstlisting}[style=bash]
# Windows
python --version
pip --version

# macOS / Linux
python3 --version
pip3 --version
        \end{lstlisting}
    \end{codebox}
    
    \begin{successbox}[Expected Output]
        \begin{lstlisting}[basicstyle=\ttfamily\small]
Python 3.11.x
pip 23.x.x
        \end{lstlisting}
    \end{successbox}
    
    \begin{warningbox}[If Not Installed]
        Download from \urlbox{https://python.org/downloads/}
        
        \textbf{Important:} Check "Add Python to PATH" during installation!
    \end{warningbox}
\end{frame}

\begin{frame}{Step 2: VS Code Setup}
    \begin{enumerate}
        \item Open VS Code
        \item Go to Extensions (Ctrl+Shift+X)
        \item Install these extensions:
    \end{enumerate}
    
    \begin{columns}[T]
        \begin{column}{0.5\textwidth}
            \begin{block}{Required}
                \begin{itemize}
                    \item Python (Microsoft)
                    \item Pylance
                \end{itemize}
            \end{block}
        \end{column}
        \begin{column}{0.45\textwidth}
            \begin{block}{Recommended}
                \begin{itemize}
                    \item Python Indent
                    \item Better Comments
                    \item Error Lens
                \end{itemize}
            \end{block}
        \end{column}
    \end{columns}
    
    \vspace{0.5cm}
    
    \begin{infobox}[Create Project Folder]
        \begin{lstlisting}[style=bash]
mkdir csi403-labs
cd csi403-labs
mkdir lab01-python
cd lab01-python
code .  # Opens VS Code in this folder
        \end{lstlisting}
    \end{infobox}
\end{frame}

% ============================================
% SECTION 2: PYTHON BASICS
% ============================================
\section{Python Basics}

\begin{frame}{Exercise 1: Hello World}
    Create file: \filepath{01\_hello.py}
    
    \begin{codebox}[01\_hello.py]
        \begin{lstlisting}[style=python]
# CSI403 Lab 1 - Exercise 1
# My first Python program

print("Hello, World!")
print("Welcome to CSI403 Full Stack Development")
print("=" * 40)
        \end{lstlisting}
    \end{codebox}
    
    \begin{block}{Run the program}
        \cmd{python 01\_hello.py}
    \end{block}
    
    \begin{exampleblock}{Expected Output}
        \begin{lstlisting}[basicstyle=\ttfamily\small]
Hello, World!
Welcome to CSI403 Full Stack Development
========================================
        \end{lstlisting}
    \end{exampleblock}
\end{frame}

\begin{frame}{Exercise 2: Variables and Data Types}
    Create file: \filepath{02\_variables.py}
    
    \begin{codebox}[02\_variables.py]
        \begin{lstlisting}[style=python]
# Variables and Data Types

# String
name = "CSI403"
instructor = "Aj. Methas"

# Integer
credits = 3
students = 45

# Float
grade_weight = 40.0

# Boolean
is_required = True

# Print all variables
print(f"Course: {name}")
print(f"Instructor: {instructor}")
print(f"Credits: {credits}")
print(f"Students: {students}")
print(f"Project Weight: {grade_weight}%")
print(f"Required Course: {is_required}")
        \end{lstlisting}
    \end{codebox}
\end{frame}

\begin{frame}{Data Types in Python}
    \begin{center}
        \begin{tabularx}{\textwidth}{lXl}
            \toprule
            \textbf{Type} & \textbf{Description} & \textbf{Example} \\
            \midrule
            \code{str} & Text/String & \code{"Hello"}, \code{'World'} \\
            \code{int} & Integer number & \code{42}, \code{-10}, \code{0} \\
            \code{float} & Decimal number & \code{3.14}, \code{-0.5} \\
            \code{bool} & Boolean & \code{True}, \code{False} \\
            \code{list} & Ordered collection & \code{[1, 2, 3]} \\
            \code{dict} & Key-value pairs & \code{\{"name": "John"\}} \\
            \code{None} & Null/Empty & \code{None} \\
            \bottomrule
        \end{tabularx}
    \end{center}
    
    \vspace{0.5cm}
    
    \begin{codebox}[Check Type]
        \begin{lstlisting}[style=python]
name = "CSI403"
print(type(name))  # <class 'str'>

credits = 3
print(type(credits))  # <class 'int'>
        \end{lstlisting}
    \end{codebox}
\end{frame}

\begin{frame}{Exercise 3: Basic Calculations}
    Create file: \filepath{03\_calculations.py}
    
    \begin{codebox}[Loan Interest Calculator]
        \begin{lstlisting}[style=python]
# Simple Loan Calculator (Preview of our project!)

# Loan information
principal = 100000  # Loan amount in THB
annual_rate = 5.5   # Interest rate (%)
years = 3           # Loan term

# Calculate simple interest
interest = principal * (annual_rate / 100) * years
total = principal + interest
monthly = total / (years * 12)

# Display results
print("=== Loan Calculator ===")
print(f"Principal: {principal:,.2f} THB")
print(f"Interest Rate: {annual_rate}%")
print(f"Term: {years} years")
print("-" * 25)
print(f"Total Interest: {interest:,.2f} THB")
print(f"Total Payment: {total:,.2f} THB")
print(f"Monthly Payment: {monthly:,.2f} THB")
        \end{lstlisting}
    \end{codebox}
\end{frame}

\begin{frame}{Python Operators}
    \begin{columns}[T]
        \begin{column}{0.48\textwidth}
            \begin{block}{Arithmetic Operators}
                \begin{tabular}{ll}
                    \code{+} & Addition \\
                    \code{-} & Subtraction \\
                    \code{*} & Multiplication \\
                    \code{/} & Division \\
                    \code{//} & Floor Division \\
                    \code{\%} & Modulus \\
                    \code{**} & Power \\
                \end{tabular}
            \end{block}
        \end{column}
        \begin{column}{0.48\textwidth}
            \begin{block}{Comparison Operators}
                \begin{tabular}{ll}
                    \code{==} & Equal \\
                    \code{!=} & Not Equal \\
                    \code{>} & Greater Than \\
                    \code{<} & Less Than \\
                    \code{>=} & Greater or Equal \\
                    \code{<=} & Less or Equal \\
                \end{tabular}
            \end{block}
        \end{column}
    \end{columns}
    
    \vspace{0.5cm}
    
    \begin{codebox}[Examples]
        \begin{lstlisting}[style=python]
print(10 / 3)   # 3.333...
print(10 // 3)  # 3 (floor division)
print(10 % 3)   # 1 (remainder)
print(2 ** 10)  # 1024 (power)
        \end{lstlisting}
    \end{codebox}
\end{frame}

% ============================================
% SECTION 3: USER INPUT
% ============================================
\section{User Input}

\begin{frame}{Exercise 4: Getting User Input}
    Create file: \filepath{04\_input.py}
    
    \begin{codebox}[User Input]
        \begin{lstlisting}[style=python]
# Getting input from user

print("=== Student Registration ===")

# Get string input
name = input("Enter your name: ")
student_id = input("Enter your student ID: ")

# Get numeric input (convert from string)
age = int(input("Enter your age: "))
gpa = float(input("Enter your GPA: "))

# Display information
print("\n=== Registration Complete ===")
print(f"Name: {name}")
print(f"Student ID: {student_id}")
print(f"Age: {age}")
print(f"GPA: {gpa:.2f}")
        \end{lstlisting}
    \end{codebox}
    
    \begin{warningbox}[Note]
        \code{input()} always returns a string. Use \code{int()} or \code{float()} to convert!
    \end{warningbox}
\end{frame}

\begin{frame}{Exercise 5: Loan Application Form}
    Create file: \filepath{05\_loan\_form.py}
    
    \begin{codebox}[Loan Application]
        \begin{lstlisting}[style=python]
# Loan Application Form (Mini Version)

print("=" * 40)
print("   LOAN APPLICATION FORM")
print("=" * 40)

# Borrower Information
name = input("Full Name: ")
income = float(input("Annual Income (THB): "))

# Loan Request
amount = float(input("Loan Amount (THB): "))
term = int(input("Loan Term (months): "))

# Simple calculation
rate = 7.5  # Fixed rate for now
monthly_payment = (amount * (1 + rate/100 * term/12)) / term

print("\n" + "=" * 40)
print("   APPLICATION SUMMARY")
print("=" * 40)
print(f"Applicant: {name}")
print(f"Requested Amount: {amount:,.2f} THB")
print(f"Term: {term} months")
print(f"Estimated Monthly Payment: {monthly_payment:,.2f} THB")
        \end{lstlisting}
    \end{codebox}
\end{frame}

% ============================================
% SECTION 4: STRINGS
% ============================================
\section{Working with Strings}

\begin{frame}{String Operations}
    \begin{codebox}[String Methods]
        \begin{lstlisting}[style=python]
name = "  John Smith  "

# Common string methods
print(name.strip())       # "John Smith" (remove spaces)
print(name.lower())       # "  john smith  "
print(name.upper())       # "  JOHN SMITH  "
print(name.title())       # "  John Smith  "
print(name.replace("John", "Jane"))

# String formatting
loan_id = "LN"
number = 1234
formatted = f"{loan_id}-{number:06d}"  # "LN-001234"
print(formatted)

# String operations
print(len(name))          # 14 (including spaces)
print("John" in name)     # True
print(name.split())       # ['John', 'Smith']
        \end{lstlisting}
    \end{codebox}
\end{frame}

\begin{frame}{Exercise 6: String Formatting}
    Create file: \filepath{06\_strings.py}
    
    \begin{codebox}[Format Loan ID]
        \begin{lstlisting}[style=python]
# Generate formatted loan IDs

# Get user input
year = input("Year (2 digits): ")
month = input("Month (2 digits): ")
sequence = int(input("Sequence number: "))

# Format: LN-YYMM-XXXXX
loan_id = f"LN-{year}{month}-{sequence:05d}"

print(f"\nGenerated Loan ID: {loan_id}")

# Example output: LN-2601-00001

# Bonus: Generate multiple IDs
print("\nSample Loan IDs:")
for i in range(1, 6):
    sample_id = f"LN-{year}{month}-{i:05d}"
    print(f"  {sample_id}")
        \end{lstlisting}
    \end{codebox}
\end{frame}

% ============================================
% SECTION 5: ASSIGNMENT
% ============================================
\section{Lab Assignment}

\begin{frame}{Lab 1 Assignment}
    \begin{alertblock}{Task: Enhanced Loan Calculator}
        Create \filepath{assignment\_loan\_calc.py} that:
        \begin{enumerate}
            \item Asks user for their name
            \item Asks for loan amount (validate > 0)
            \item Asks for interest rate
            \item Asks for loan term in years
            \item Calculates and displays:
            \begin{itemize}
                \item Total interest
                \item Total payment
                \item Monthly payment
            \end{itemize}
            \item Generates a loan ID
        \end{enumerate}
    \end{alertblock}
\end{frame}

\begin{frame}{Expected Output}
    \begin{codebox}[Sample Output]
        \begin{lstlisting}[basicstyle=\ttfamily\small]
========================================
       LOAN CALCULATOR v1.0
========================================
Enter your name: John Doe
Enter loan amount (THB): 500000
Enter annual interest rate (%): 6.5
Enter loan term (years): 5

========================================
       LOAN SUMMARY
========================================
Loan ID: LN-2601-00001
Applicant: John Doe
----------------------------------------
Principal Amount: 500,000.00 THB
Interest Rate: 6.5%
Loan Term: 5 years (60 months)
----------------------------------------
Total Interest: 162,500.00 THB
Total Payment: 662,500.00 THB
Monthly Payment: 11,041.67 THB
========================================
        \end{lstlisting}
    \end{codebox}
\end{frame}

\begin{frame}{Submission Instructions}
    \begin{enumerate}
        \item Create all exercise files (01-06) + assignment
        \item Push to your GitHub repository:
    \end{enumerate}
    
    \begin{codebox}[Git Commands]
        \begin{lstlisting}[style=bash]
# Add all files
git add .

# Commit with message
git commit -m "Lab 1: Python basics completed"

# Push to GitHub
git push origin main
        \end{lstlisting}
    \end{codebox}
    
    \begin{block}{Grading Criteria (2\%)}
        \begin{itemize}
            \item All exercises completed: 1\%
            \item Assignment works correctly: 0.5\%
            \item Clean code \& comments: 0.5\%
        \end{itemize}
    \end{block}
    
    \begin{center}
        \fbox{\large Deadline: Before next class (Jan 14)}
    \end{center}
\end{frame}

% ============================================
% REFERENCE CARD
% ============================================
\begin{frame}{Python Quick Reference}
    \small
    \begin{columns}[T]
        \begin{column}{0.48\textwidth}
            \begin{block}{Variables}
                \begin{lstlisting}[style=python,basicstyle=\ttfamily\tiny]
name = "text"    # string
age = 25         # int
rate = 5.5       # float
is_ok = True     # bool
                \end{lstlisting}
            \end{block}
            
            \begin{block}{Input/Output}
                \begin{lstlisting}[style=python,basicstyle=\ttfamily\tiny]
# Input
x = input("prompt: ")
n = int(input("number: "))

# Output
print("Hello")
print(f"Value: {x}")
                \end{lstlisting}
            \end{block}
        \end{column}
        \begin{column}{0.48\textwidth}
            \begin{block}{Formatting}
                \begin{lstlisting}[style=python,basicstyle=\ttfamily\tiny]
# f-string formatting
f"{name}"         # variable
f"{num:.2f}"      # 2 decimals
f"{num:,.2f}"     # with comma
f"{num:05d}"      # zero pad
                \end{lstlisting}
            \end{block}
            
            \begin{block}{String Methods}
                \begin{lstlisting}[style=python,basicstyle=\ttfamily\tiny]
s.strip()    # remove spaces
s.lower()    # lowercase
s.upper()    # uppercase
s.split()    # split to list
len(s)       # length
                \end{lstlisting}
            \end{block}
        \end{column}
    \end{columns}
\end{frame}

% ============================================
% FINAL SLIDE
% ============================================
\begin{frame}[plain]
    \begin{center}
        \vspace{2cm}
        {\Huge\bfseries\color{spumaroon} Lab 1 Complete!}
        
        \vspace{1cm}
        
        {\Large See you next week!}
        
        \vspace{0.5cm}
        
        Next Lab: HTML \& CSS Basics
        
        \vspace{1cm}
        
        \faQuestionCircle\ Questions? Ask now or on GitHub Issues
        
        \vspace{1cm}
        
        \includegraphics[height=1cm]{spu_logo.png}
    \end{center}
\end{frame}

\end{document}
