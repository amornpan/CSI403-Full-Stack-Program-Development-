% ============================================
% Week 1: Course Introduction & Full Stack Overview
% CSI403 Full Stack Development
% Date: January 7-9, 2026
% ============================================

% ============================================
% CSI403 Full Stack Development
% Common Preamble for All Presentations
% SPU Chonburi - Semester 2/2568
% ============================================

\documentclass[aspectratio=169,11pt]{beamer}

% ============================================
% PACKAGES
% ============================================
% Theme
\usepackage{../common/beamerthemeSPU}

% Graphics
\usepackage{graphicx}
\graphicspath{{./images/}{../common/images/}}

% TikZ for diagrams
\usepackage{tikz}
\usetikzlibrary{shapes,arrows,positioning,fit,calc,shadows,decorations.pathreplacing}

% Tables
\usepackage{booktabs}
\usepackage{tabularx}
\usepackage{multirow}
\usepackage{colortbl}

% Code listings
\usepackage{listings}
\usepackage{minted}
\usemintedstyle{friendly}

% Boxes
\usepackage{tcolorbox}
\tcbuselibrary{skins,breakable}

% Math
\usepackage{amsmath}
\usepackage{amssymb}

% Icons
\usepackage{fontawesome5}

% Hyperlinks
\usepackage{hyperref}
\hypersetup{
    colorlinks=true,
    linkcolor=spumaroon,
    urlcolor=codeblue,
    citecolor=spumaroon
}

% ============================================
% LISTINGS SETTINGS
% ============================================
\lstset{
    basicstyle=\ttfamily\small,
    backgroundcolor=\color{codebackground},
    keywordstyle=\color{codeblue}\bfseries,
    stringstyle=\color{codegreen},
    commentstyle=\color{codegray}\itshape,
    numberstyle=\tiny\color{codegray},
    numbers=left,
    numbersep=5pt,
    breaklines=true,
    breakatwhitespace=true,
    tabsize=4,
    showspaces=false,
    showstringspaces=false,
    frame=single,
    framerule=0.5pt,
    rulecolor=\color{spulightmaroon},
    xleftmargin=15pt,
    framexleftmargin=15pt,
    aboveskip=10pt,
    belowskip=10pt
}

% Python style
\lstdefinestyle{python}{
    language=Python,
    morekeywords={self,True,False,None,as,with,yield,async,await},
    morestring=[b]""",
    morestring=[b]'''
}

% HTML style
\lstdefinestyle{html}{
    language=HTML,
    morekeywords={DOCTYPE,html,head,body,div,span,class,id,href,src}
}

% SQL style
\lstdefinestyle{sql}{
    language=SQL,
    morekeywords={INT,VARCHAR,DECIMAL,DATETIME,PRIMARY,KEY,FOREIGN,REFERENCES,
                  CREATE,TABLE,INSERT,INTO,VALUES,SELECT,FROM,WHERE,JOIN,ON,
                  UPDATE,SET,DELETE,DROP,ALTER,ADD,INDEX,CONSTRAINT}
}

% Bash style
\lstdefinestyle{bash}{
    language=bash,
    morekeywords={docker,docker-compose,pip,python,pytest,git,curl,cd,ls,mkdir}
}

% ============================================
% TCOLORBOX STYLES
% ============================================
\newtcolorbox{codebox}[1][]{
    colback=codebackground,
    colframe=spumaroon,
    boxrule=1pt,
    arc=3pt,
    fonttitle=\bfseries,
    title=#1
}

\newtcolorbox{infobox}[1][]{
    colback=spulightgray,
    colframe=codeblue,
    boxrule=1pt,
    arc=3pt,
    fonttitle=\bfseries,
    title=#1
}

\newtcolorbox{warningbox}[1][]{
    colback=orange!10,
    colframe=codeorange,
    boxrule=1pt,
    arc=3pt,
    fonttitle=\bfseries,
    title=#1
}

\newtcolorbox{successbox}[1][]{
    colback=green!10,
    colframe=codegreen,
    boxrule=1pt,
    arc=3pt,
    fonttitle=\bfseries,
    title=#1
}

% ============================================
% CUSTOM COMMANDS
% ============================================
% File path
\newcommand{\filepath}[1]{\texttt{\color{codepurple}#1}}

% Terminal command
\newcommand{\cmd}[1]{\texttt{\color{codegreen}\$ #1}}

% URL display
\newcommand{\urlbox}[1]{\fcolorbox{codeblue}{codebackground}{\texttt{\color{codeblue}#1}}}

% Keyboard key
\newcommand{\key}[1]{\fbox{\small\texttt{#1}}}

% Tech logo placeholder
\newcommand{\techlogo}[2][1cm]{\includegraphics[height=#1]{tech-icons/#2}}

% Week indicator
\newcommand{\weekheader}[2]{%
    \begin{tikzpicture}[remember picture,overlay]
        \node[anchor=north east,xshift=-0.5cm,yshift=-0.3cm,
              fill=spugold,text=spumaroon,font=\small\bfseries,
              rounded corners=3pt,inner sep=5pt] 
              at (current page.north east) {Week #1 | #2};
    \end{tikzpicture}
}

% Progress bar
\newcommand{\progressbar}[2]{%
    \begin{tikzpicture}
        \fill[spulightgray] (0,0) rectangle (10,0.3);
        \fill[spumaroon] (0,0) rectangle (#1,0.3);
        \node[anchor=west] at (10.2,0.15) {\small #2};
    \end{tikzpicture}
}

% ============================================
% TITLE PAGE LOGO
% ============================================
\titlegraphic{%
    \includegraphics[height=1.5cm]{spu_logo.png}
}

% ============================================
% COURSE INFORMATION
% ============================================
\newcommand{\coursetitle}{CSI403: Full Stack Development}
\newcommand{\coursesemester}{Semester 2/2568}
\newcommand{\instructor}{Aj. Methas Khamjad}
\newcommand{\university}{Sripatum University Chonburi}

% Default author/institute
\author[\instructor]{\instructor}
\institute[\university]{%
    School of Information Technology\\
    \university
}

\endinput


\title[Week 1: Introduction]{Week 1: Course Introduction \& Full Stack Overview}
\subtitle{CSI403 Full Stack Development}
\date{January 7-9, 2026}

\begin{document}

% ============================================
% TITLE SLIDE
% ============================================
\begin{frame}[plain]
    \titlepage
\end{frame}

% ============================================
% OUTLINE
% ============================================
\begin{frame}{Today's Agenda}
    \tableofcontents
\end{frame}

% ============================================
% SECTION 1: WELCOME
% ============================================
\section{Welcome to CSI403}

\begin{frame}{Welcome to Full Stack Development!}
    \weekheader{1}{Jan 7-9}
    
    \begin{columns}[T]
        \begin{column}{0.6\textwidth}
            \begin{block}{Course Information}
                \begin{itemize}
                    \item \textbf{Course Code:} CSI403
                    \item \textbf{Credits:} 3 (2-3-5)
                    \item \textbf{Semester:} 2/2568
                    \item \textbf{Duration:} 15 weeks
                    \item \textbf{Schedule:} Wed \& Fri
                \end{itemize}
            \end{block}
        \end{column}
        \begin{column}{0.35\textwidth}
            \begin{block}{Instructor}
                \centering
                \faUserTie\\[5pt]
                \textbf{Aj. Methas Khamjad}\\
                School of IT\\
                SPU Chonburi
            \end{block}
        \end{column}
    \end{columns}
\end{frame}

\begin{frame}{Course Schedule Overview}
    \begin{center}
        \begin{tikzpicture}[scale=0.9]
            % Timeline
            \draw[thick,spumaroon] (0,0) -- (14,0);
            
            % Week markers
            \foreach \x in {0,1,...,14} {
                \draw[spumaroon] (\x,0.1) -- (\x,-0.1);
            }
            
            % Phase boxes
            \fill[spumaroon!20] (0,-0.5) rectangle (2.8,-1.2);
            \node at (1.4,-0.85) {\tiny\textbf{Foundation}};
            
            \fill[codeblue!20] (3,-0.5) rectangle (5.8,-1.2);
            \node at (4.4,-0.85) {\tiny\textbf{Design}};
            
            \fill[codegreen!20] (6,-0.5) rectangle (8.8,-1.2);
            \node at (7.4,-0.85) {\tiny\textbf{Development}};
            
            \fill[codeorange!20] (9,-0.5) rectangle (10.8,-1.2);
            \node at (9.9,-0.85) {\tiny\textbf{DevOps}};
            
            \fill[codepurple!20] (11,-0.5) rectangle (14,-1.2);
            \node at (12.5,-0.85) {\tiny\textbf{Testing \& Final}};
            
            % Labels
            \node[above] at (0,0.2) {\tiny Wk1};
            \node[above] at (7,0.2) {\tiny Wk8};
            \node[above] at (14,0.2) {\tiny Wk15};
            
            % Current position
            \fill[spumaroon] (0,0) circle (4pt);
            \node[below] at (0,-1.5) {\small\textbf{You are here!}};
        \end{tikzpicture}
    \end{center}
    
    \vspace{0.5cm}
    
    \begin{columns}[T]
        \begin{column}{0.45\textwidth}
            \textbf{Key Dates:}
            \begin{itemize}
                \item Start: Jan 7, 2026
                \item End: Apr 24, 2026
                \item Songkran Break: Apr 13-15
            \end{itemize}
        \end{column}
        \begin{column}{0.45\textwidth}
            \textbf{Class Days:}
            \begin{itemize}
                \item Wednesday (Lecture)
                \item Friday (Lab)
            \end{itemize}
        \end{column}
    \end{columns}
\end{frame}

\begin{frame}{Grading Criteria}
    \begin{center}
        \begin{tikzpicture}
            % Pie chart
            \pie[
                text=legend,
                radius=2.5,
                color={spumaroon, codeblue, codegreen, codeorange, codepurple, codegray}
            ]{
                40/Project,
                20/Exercises,
                10/Quiz,
                10/Presentation,
                10/Workpiece,
                10/Attendance
            }
        \end{tikzpicture}
    \end{center}
    
    \begin{alertblock}{Important}
        \textbf{Project (40\%)} is the core assessment - a complete Loan Management System!
    \end{alertblock}
\end{frame}

\begin{frame}{Assessment Breakdown}
    \small
    \begin{tabularx}{\textwidth}{lXcl}
        \toprule
        \textbf{Item} & \textbf{Description} & \textbf{Weight} & \textbf{Due} \\
        \midrule
        Project & Loan Management System & 40\% & Week 15 \\
        \midrule
        G2\_PM & SRS + Notion Board & 5\% & Week 3 \\
        Lab1\_Design & UI Mockup + Wireframe & 5\% & Week 5 \\
        G3\_Design & ERD + API + UI & 5\% & Week 7 \\
        P1\_Diagram & Flowcharts + DFD & 5\% & Week 9 \\
        P2\_TestDoc & Test Plan + Matrix & 5\% & Week 13 \\
        \midrule
        Q1\_CI/CD & Docker, Jenkins Quiz & 5\% & Week 9 \\
        Q2\_Testing & Testing Methods Quiz & 5\% & Week 11 \\
        \midrule
        Workpiece & System Demo & 10\% & Week 14-15 \\
        Attendance & Class + Lab & 10\% & Weekly \\
        \bottomrule
    \end{tabularx}
\end{frame}

% ============================================
% SECTION 2: WHAT IS FULL STACK?
% ============================================
\section{What is Full Stack Development?}

\sectionslide{What is Full Stack?}

\begin{frame}{The Full Stack Concept}
    \begin{center}
        \begin{tikzpicture}[
            box/.style={rectangle, rounded corners, minimum width=10cm, minimum height=1.2cm, draw, thick},
            arrow/.style={->, thick, >=stealth}
        ]
            % Frontend
            \node[box, fill=codeblue!30] (frontend) at (0,4) {
                \textbf{Frontend (Client-Side)}
            };
            \node[right] at (5.5,4) {\small HTML, CSS, JavaScript, Jinja2};
            
            % Backend
            \node[box, fill=codegreen!30] (backend) at (0,2) {
                \textbf{Backend (Server-Side)}
            };
            \node[right] at (5.5,2) {\small FastAPI, Python, Business Logic};
            
            % Database
            \node[box, fill=codeorange!30] (database) at (0,0) {
                \textbf{Database}
            };
            \node[right] at (5.5,0) {\small MSSQL, SQLAlchemy ORM};
            
            % Arrows
            \draw[arrow, spumaroon] (frontend) -- (backend);
            \draw[arrow, spumaroon] (backend) -- (database);
            \draw[arrow, spumaroon] (backend) -- (frontend);
            \draw[arrow, spumaroon] (database) -- (backend);
            
            % Labels
            \node[left, text=spumaroon] at (-5.5,3) {\small HTTP Request};
            \node[left, text=spumaroon] at (-5.5,1) {\small SQL Query};
        \end{tikzpicture}
    \end{center}
\end{frame}

\begin{frame}{Our Technology Stack}
    \begin{center}
        \begin{tikzpicture}[
            layer/.style={rectangle, rounded corners, minimum width=3cm, minimum height=1cm, draw, thick, font=\small}
        ]
            % Frontend Stack
            \node[layer, fill=codeblue!20] (jinja) at (0,3) {Jinja2};
            \node[layer, fill=codeblue!20] (html) at (0,2) {HTML5/CSS3};
            \node[layer, fill=codeblue!20] (bootstrap) at (0,1) {Bootstrap 5};
            \node[below, font=\bfseries] at (0,0.3) {Frontend};
            
            % Backend Stack
            \node[layer, fill=codegreen!20] (fastapi) at (4,3) {FastAPI};
            \node[layer, fill=codegreen!20] (pydantic) at (4,2) {Pydantic};
            \node[layer, fill=codegreen!20] (sqlalchemy) at (4,1) {SQLAlchemy};
            \node[below, font=\bfseries] at (4,0.3) {Backend};
            
            % Database Stack
            \node[layer, fill=codeorange!20] (mssql) at (8,3) {MSSQL};
            \node[layer, fill=codeorange!20] (docker) at (8,2) {Docker};
            \node[layer, fill=codeorange!20] (jenkins) at (8,1) {Jenkins};
            \node[below, font=\bfseries] at (8,0.3) {DevOps};
            
            % Tools Stack
            \node[layer, fill=codepurple!20] (github) at (12,3) {GitHub};
            \node[layer, fill=codepurple!20] (notion) at (12,2) {Notion};
            \node[layer, fill=codepurple!20] (vscode) at (12,1) {VS Code};
            \node[below, font=\bfseries] at (12,0.3) {Tools};
        \end{tikzpicture}
    \end{center}
    
    \begin{infobox}[Why This Stack?]
        \begin{itemize}
            \item \textbf{Industry Standard} - Used by real companies
            \item \textbf{Beginner Friendly} - Easy to learn, powerful to use
            \item \textbf{Complete Workflow} - From code to deployment
        \end{itemize}
    \end{infobox}
\end{frame}

\begin{frame}{Frontend: What Users See}
    \begin{columns}[T]
        \begin{column}{0.5\textwidth}
            \begin{block}{Technologies}
                \begin{itemize}
                    \item \textbf{Jinja2} - Template Engine
                    \item \textbf{HTML5} - Structure
                    \item \textbf{CSS3} - Styling
                    \item \textbf{Bootstrap 5} - UI Components
                \end{itemize}
            \end{block}
            
            \vspace{0.5cm}
            
            \begin{exampleblock}{Approach}
                Server-Side Rendering (SSR)\\
                \small Templates rendered on server, sent as HTML to browser
            \end{exampleblock}
        \end{column}
        \begin{column}{0.45\textwidth}
            \begin{codebox}[Jinja2 Example]
                \begin{lstlisting}[style=html,basicstyle=\ttfamily\tiny]


<h1>Welcome, {{ user.name }}!</h1>
<ul>

    <li>{{ loan.amount | currency }}</li>

</ul>

                \end{lstlisting}
            \end{codebox}
        \end{column}
    \end{columns}
\end{frame}

\begin{frame}{Backend: The Brain}
    \begin{columns}[T]
        \begin{column}{0.5\textwidth}
            \begin{block}{Technologies}
                \begin{itemize}
                    \item \textbf{FastAPI} - Web Framework
                    \item \textbf{Python} - Programming Language
                    \item \textbf{Pydantic} - Data Validation
                    \item \textbf{SQLAlchemy} - ORM
                \end{itemize}
            \end{block}
            
            \vspace{0.5cm}
            
            \begin{alertblock}{Key Responsibilities}
                \begin{itemize}
                    \item Handle HTTP requests
                    \item Business logic
                    \item Database operations
                    \item Authentication
                \end{itemize}
            \end{alertblock}
        \end{column}
        \begin{column}{0.45\textwidth}
            \begin{codebox}[FastAPI Example]
                \begin{lstlisting}[style=python,basicstyle=\ttfamily\tiny]
from fastapi import FastAPI

app = FastAPI()

@app.get("/loans")
def get_loans():
    return {"loans": [...]}

@app.post("/loans")
def create_loan(loan: LoanCreate):
    # Business logic here
    return {"id": 1, ...}
                \end{lstlisting}
            \end{codebox}
        \end{column}
    \end{columns}
\end{frame}

\begin{frame}{Database: Data Storage}
    \begin{columns}[T]
        \begin{column}{0.5\textwidth}
            \begin{block}{Our Choice: MSSQL}
                \begin{itemize}
                    \item \textbf{Why MSSQL?}
                    \begin{itemize}
                        \item Industry standard
                        \item Enterprise features
                        \item Thai company preference
                    \end{itemize}
                    \item \textbf{Running in Docker}
                    \begin{itemize}
                        \item Easy setup
                        \item Consistent environment
                        \item No installation hassle
                    \end{itemize}
                \end{itemize}
            \end{block}
        \end{column}
        \begin{column}{0.45\textwidth}
            \begin{codebox}[SQL Example]
                \begin{lstlisting}[style=sql,basicstyle=\ttfamily\tiny]
CREATE TABLE loans (
    id INT PRIMARY KEY,
    borrower_id INT,
    loan_amount DECIMAL(15,2),
    interest_rate DECIMAL(5,2),
    status VARCHAR(20),
    FOREIGN KEY (borrower_id)
        REFERENCES borrowers(id)
);
                \end{lstlisting}
            \end{codebox}
        \end{column}
    \end{columns}
\end{frame}

% ============================================
% SECTION 3: DEVOPS & CI/CD
% ============================================
\section{DevOps \& CI/CD Introduction}

\sectionslide{DevOps \& CI/CD}

\begin{frame}{What is DevOps?}
    \begin{center}
        \begin{tikzpicture}
            % Infinity loop
            \draw[ultra thick, spumaroon, ->] (0,0) arc (180:0:2 and 1);
            \draw[ultra thick, codeblue, ->] (4,0) arc (0:-180:2 and 1);
            
            % Labels
            \node[above, font=\bfseries] at (2,1.2) {DEVELOPMENT};
            \node[below, font=\bfseries] at (2,-1.2) {OPERATIONS};
            
            % Steps - Dev side
            \node at (-1,0.3) {\small Plan};
            \node at (0.5,0.8) {\small Code};
            \node at (2,1) {\small Build};
            \node at (3.5,0.8) {\small Test};
            
            % Steps - Ops side
            \node at (5,0.3) {\small Release};
            \node at (3.5,-0.8) {\small Deploy};
            \node at (2,-1) {\small Operate};
            \node at (0.5,-0.8) {\small Monitor};
            
            % Center
            \node[circle, draw, thick, fill=spugold!30, minimum size=1.5cm] at (2,0) {\textbf{DevOps}};
        \end{tikzpicture}
    \end{center}
    
    \vspace{0.5cm}
    
    \begin{columns}[T]
        \begin{column}{0.45\textwidth}
            \begin{block}{Goals}
                \begin{itemize}
                    \item Faster delivery
                    \item Better quality
                    \item Team collaboration
                \end{itemize}
            \end{block}
        \end{column}
        \begin{column}{0.45\textwidth}
            \begin{block}{Our Tools}
                \begin{itemize}
                    \item Docker (Containers)
                    \item Jenkins (Automation)
                    \item GitHub (Version Control)
                \end{itemize}
            \end{block}
        \end{column}
    \end{columns}
\end{frame}

\begin{frame}{CI/CD Pipeline}
    \begin{center}
        \begin{tikzpicture}[
            stage/.style={rectangle, rounded corners, minimum width=2cm, minimum height=1.5cm, draw, thick, align=center},
            arrow/.style={->, thick, >=stealth}
        ]
            % Stages
            \node[stage, fill=codeblue!20] (code) at (0,0) {\faCode\\Code};
            \node[stage, fill=codegreen!20] (build) at (3,0) {\faCogs\\Build};
            \node[stage, fill=codeorange!20] (test) at (6,0) {\faVial\\Test};
            \node[stage, fill=codepurple!20] (deploy) at (9,0) {\faRocket\\Deploy};
            \node[stage, fill=spumaroon!20] (prod) at (12,0) {\faGlobe\\Production};
            
            % Arrows
            \draw[arrow] (code) -- (build);
            \draw[arrow] (build) -- (test);
            \draw[arrow] (test) -- (deploy);
            \draw[arrow] (deploy) -- (prod);
            
            % Labels below
            \node[below, font=\tiny] at (0,-1) {Git Push};
            \node[below, font=\tiny] at (3,-1) {Docker Build};
            \node[below, font=\tiny] at (6,-1) {pytest};
            \node[below, font=\tiny] at (9,-1) {Jenkins};
            \node[below, font=\tiny] at (12,-1) {Live!};
            
            % CI/CD labels
            \draw[decorate, decoration={brace, amplitude=10pt, raise=5pt}] 
                (0,1) -- (6,1) node[midway, above=15pt] {\textbf{CI} - Continuous Integration};
            \draw[decorate, decoration={brace, amplitude=10pt, raise=5pt}] 
                (6,1) -- (12,1) node[midway, above=15pt] {\textbf{CD} - Continuous Deployment};
        \end{tikzpicture}
    \end{center}
    
    \vspace{0.5cm}
    
    \begin{successbox}[Benefits]
        \begin{itemize}
            \item \textbf{Automated} - No manual deployment
            \item \textbf{Consistent} - Same process every time
            \item \textbf{Fast Feedback} - Know if code breaks immediately
        \end{itemize}
    \end{successbox}
\end{frame}

\begin{frame}{Docker: Containerization}
    \begin{columns}[T]
        \begin{column}{0.55\textwidth}
            \begin{block}{What is Docker?}
                Package your application with all dependencies into a standardized unit called a \textbf{container}.
            \end{block}
            
            \vspace{0.3cm}
            
            \begin{alertblock}{Why Docker?}
                \begin{itemize}
                    \item "Works on my machine" → Works everywhere!
                    \item Easy to share environment
                    \item Isolate applications
                    \item Quick setup
                \end{itemize}
            \end{alertblock}
        \end{column}
        \begin{column}{0.4\textwidth}
            \begin{center}
                \begin{tikzpicture}[scale=0.8]
                    % Container boxes
                    \node[rectangle, draw, thick, fill=codeblue!20, minimum width=2.5cm, minimum height=1cm] (app1) at (0,3) {FastAPI App};
                    \node[rectangle, draw, thick, fill=codeorange!20, minimum width=2.5cm, minimum height=1cm] (app2) at (0,1.8) {MSSQL DB};
                    
                    % Docker layer
                    \node[rectangle, draw, thick, fill=spumaroon!20, minimum width=4cm, minimum height=0.8cm] (docker) at (0,0.5) {Docker Engine};
                    
                    % OS layer
                    \node[rectangle, draw, thick, fill=codegray!20, minimum width=4cm, minimum height=0.6cm] (os) at (0,-0.3) {Host OS};
                    
                    % Hardware
                    \node[rectangle, draw, thick, fill=spugray!30, minimum width=4cm, minimum height=0.6cm] (hw) at (0,-1) {Hardware};
                    
                    % Container border
                    \draw[dashed, thick, spumaroon] (-1.5,1.2) rectangle (1.5,3.5);
                    \node[above right, font=\tiny] at (-1.5,3.5) {Containers};
                \end{tikzpicture}
            \end{center}
        \end{column}
    \end{columns}
\end{frame}

% ============================================
% SECTION 4: CASE STUDY
% ============================================
\section{Case Study: Loan Management System}

\sectionslide{Case Study Preview}

\begin{frame}{Lending Club Loan System}
    \begin{columns}[T]
        \begin{column}{0.55\textwidth}
            \begin{block}{What We'll Build}
                A complete loan management system for a peer-to-peer lending platform.
            \end{block}
            
            \vspace{0.3cm}
            
            \textbf{Core Features:}
            \begin{itemize}
                \item \faUserPlus\ User Registration \& Login
                \item \faMoneyBillWave\ Loan Application
                \item \faChartLine\ Loan Status Tracking
                \item \faUserShield\ Role-based Access (Admin/Borrower)
                \item \faHistory\ Payment History
            \end{itemize}
        \end{column}
        \begin{column}{0.4\textwidth}
            \begin{center}
                \begin{tikzpicture}[scale=0.7]
                    % Loan status flow
                    \node[rectangle, rounded corners, draw, thick, fill=codegreen!30, minimum width=2cm] (current) at (0,4) {Current};
                    \node[rectangle, rounded corners, draw, thick, fill=codeblue!30, minimum width=2cm] (paid) at (-2,2) {Fully Paid};
                    \node[rectangle, rounded corners, draw, thick, fill=codeorange!30, minimum width=2cm] (late) at (2,2) {Late};
                    \node[rectangle, rounded corners, draw, thick, fill=spumaroon!30, minimum width=2cm] (default) at (2,0) {Default};
                    
                    % Arrows
                    \draw[->, thick] (current) -- (paid);
                    \draw[->, thick] (current) -- (late);
                    \draw[->, thick] (late) -- (default);
                    \draw[->, thick, dashed] (late) -- (current);
                \end{tikzpicture}
            \end{center}
            \small\centering Loan Status Flow
        \end{column}
    \end{columns}
\end{frame}

\begin{frame}{Database Design Preview}
    \begin{center}
        \begin{tikzpicture}[
            table/.style={rectangle, draw, thick, minimum width=3cm, minimum height=2cm, align=left, font=\tiny},
            pk/.style={font=\tiny\bfseries, text=spumaroon},
            fk/.style={font=\tiny, text=codeblue}
        ]
            % Users table
            \node[table, fill=spulightgray] (users) at (0,0) {
                \textbf{users}\\[3pt]
                \pk{id} (PK)\\
                username\\
                email\\
                password\_hash\\
                role\\
                is\_active
            };
            
            % Borrowers table
            \node[table, fill=spulightgray] (borrowers) at (4,0) {
                \textbf{borrowers}\\[3pt]
                \pk{id} (PK)\\
                \fk{user\_id} (FK)\\
                emp\_title\\
                annual\_inc\\
                grade
            };
            
            % Loans table
            \node[table, fill=spulightgray] (loans) at (8,0) {
                \textbf{loans}\\[3pt]
                \pk{id} (PK)\\
                \fk{borrower\_id} (FK)\\
                loan\_amnt\\
                int\_rate\\
                status
            };
            
            % Payments table
            \node[table, fill=spulightgray] (payments) at (12,0) {
                \textbf{payments}\\[3pt]
                \pk{id} (PK)\\
                \fk{loan\_id} (FK)\\
                payment\_date\\
                amount\\
                remaining
            };
            
            % Relationships
            \draw[->, thick, codeblue] (users) -- (borrowers) node[midway, above, font=\tiny] {1:1};
            \draw[->, thick, codeblue] (borrowers) -- (loans) node[midway, above, font=\tiny] {1:N};
            \draw[->, thick, codeblue] (loans) -- (payments) node[midway, above, font=\tiny] {1:N};
        \end{tikzpicture}
    \end{center}
\end{frame}

% ============================================
% SECTION 5: GETTING STARTED
% ============================================
\section{Getting Started}

\begin{frame}{Tools to Install}
    \begin{columns}[T]
        \begin{column}{0.48\textwidth}
            \begin{block}{Required Software}
                \begin{enumerate}
                    \item \textbf{Python 3.11+}
                    \begin{itemize}
                        \item \tiny \url{python.org}
                    \end{itemize}
                    \item \textbf{VS Code}
                    \begin{itemize}
                        \item \tiny \url{code.visualstudio.com}
                    \end{itemize}
                    \item \textbf{Git}
                    \begin{itemize}
                        \item \tiny \url{git-scm.com}
                    \end{itemize}
                    \item \textbf{Docker Desktop}
                    \begin{itemize}
                        \item \tiny \url{docker.com}
                    \end{itemize}
                \end{enumerate}
            \end{block}
        \end{column}
        \begin{column}{0.48\textwidth}
            \begin{block}{VS Code Extensions}
                \begin{itemize}
                    \item Python
                    \item Pylance
                    \item Docker
                    \item GitLens
                    \item Better Jinja
                    \item Thunder Client (API testing)
                \end{itemize}
            \end{block}
            
            \vspace{0.3cm}
            
            \begin{exampleblock}{Accounts Needed}
                \begin{itemize}
                    \item GitHub account
                    \item Notion account
                \end{itemize}
            \end{exampleblock}
        \end{column}
    \end{columns}
\end{frame}

\begin{frame}{Verify Your Installation}
    \begin{codebox}[Open Terminal / Command Prompt]
        \begin{lstlisting}[style=bash,basicstyle=\ttfamily\small]
# Check Python version
python --version
# Expected: Python 3.11.x or higher

# Check pip
pip --version

# Check Git
git --version

# Check Docker
docker --version
docker-compose --version
        \end{lstlisting}
    \end{codebox}
    
    \begin{warningbox}[Troubleshooting]
        If any command fails:
        \begin{itemize}
            \item Restart your terminal
            \item Check if added to PATH
            \item Reinstall the software
        \end{itemize}
    \end{warningbox}
\end{frame}

\begin{frame}{Your First Python Program}
    \begin{codebox}[hello.py]
        \begin{lstlisting}[style=python]
# CSI403 - Week 1
# My first Python program

print("Hello, Full Stack World!")

# Variables
name = "CSI403"
credits = 3

print(f"Welcome to {name}!")
print(f"This course is worth {credits} credits.")
        \end{lstlisting}
    \end{codebox}
    
    \begin{successbox}[Run It!]
        \cmd{python hello.py}
    \end{successbox}
\end{frame}

% ============================================
% SECTION 6: GROUP FORMATION
% ============================================
\section{Group Formation}

\begin{frame}{Form Your Project Team}
    \begin{columns}[T]
        \begin{column}{0.5\textwidth}
            \begin{block}{Team Requirements}
                \begin{itemize}
                    \item \textbf{Size:} 4-5 students
                    \item \textbf{Leader:} 1 person
                    \item \textbf{Roles:}
                    \begin{itemize}
                        \item Frontend Developer
                        \item Backend Developer
                        \item Database Admin
                        \item DevOps/Tester
                        \item Documentation
                    \end{itemize}
                \end{itemize}
            \end{block}
        \end{column}
        \begin{column}{0.45\textwidth}
            \begin{alertblock}{Today's Task}
                \begin{enumerate}
                    \item Find your teammates
                    \item Choose a team leader
                    \item Create team name
                    \item Create GitHub Organization
                    \item Submit team info
                \end{enumerate}
            \end{alertblock}
        \end{column}
    \end{columns}
    
    \vspace{0.5cm}
    
    \begin{center}
        \fbox{\large Deadline: End of Lab Session Today}
    \end{center}
\end{frame}

\begin{frame}{GitHub Repository Setup}
    \begin{enumerate}
        \item Go to \urlbox{github.com}
        \item Create Organization: \code{csi403-team-[name]}
        \item Create Repository: \code{loan-management-system}
        \item Add all team members as collaborators
        \item Initialize with README.md
    \end{enumerate}
    
    \vspace{0.5cm}
    
    \begin{codebox}[Clone Your Repository]
        \begin{lstlisting}[style=bash]
# Clone the repository
git clone https://github.com/csi403-team-xxx/loan-management-system.git

# Navigate into directory
cd loan-management-system

# Check status
git status
        \end{lstlisting}
    \end{codebox}
\end{frame}

% ============================================
% SUMMARY & NEXT WEEK
% ============================================
\section{Summary \& What's Next}

\begin{frame}{Week 1 Summary}
    \begin{columns}[T]
        \begin{column}{0.48\textwidth}
            \begin{block}{What We Learned}
                \begin{itemize}
                    \item[$\checkmark$] Course overview \& grading
                    \item[$\checkmark$] Full Stack concept
                    \item[$\checkmark$] Our technology stack
                    \item[$\checkmark$] DevOps \& CI/CD basics
                    \item[$\checkmark$] Case study preview
                \end{itemize}
            \end{block}
        \end{column}
        \begin{column}{0.48\textwidth}
            \begin{alertblock}{Action Items}
                \begin{itemize}
                    \item[$\square$] Install required tools
                    \item[$\square$] Create GitHub account
                    \item[$\square$] Form project team
                    \item[$\square$] Create team GitHub repo
                    \item[$\square$] Submit team info
                \end{itemize}
            \end{alertblock}
        \end{column}
    \end{columns}
\end{frame}

\begin{frame}{Next Week Preview}
    \begin{block}{Week 2: Project Planning \& SRS}
        \begin{itemize}
            \item Software Requirements Specification
            \item User Stories \& Use Cases
            \item Notion for Project Management
            \item More Python: functions, loops, lists
        \end{itemize}
    \end{block}
    
    \vspace{0.5cm}
    
    \begin{infobox}[Preparation]
        \begin{itemize}
            \item Complete tool installation
            \item Review Python basics
            \item Create Notion account
            \item Think about loan system features
        \end{itemize}
    \end{infobox}
\end{frame}

% ============================================
% FINAL SLIDE
% ============================================
\begin{frame}[plain]
    \begin{center}
        \vspace{2cm}
        {\Huge\bfseries\color{spumaroon} Questions?}
        
        \vspace{1cm}
        
        {\Large Thank you!}
        
        \vspace{1cm}
        
        \faEnvelope\ methas@spuchonburi.ac.th\\[5pt]
        \faGithub\ github.com/csi403-fullstack\\[5pt]
        \faBook\ Course Materials: Notion
        
        \vspace{1cm}
        
        \includegraphics[height=1cm]{spu_logo.png}
    \end{center}
\end{frame}

\end{document}
