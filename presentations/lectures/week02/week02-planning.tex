% ============================================
% Week 2: Project Planning & SRS
% CSI403 Full Stack Development
% Date: January 14-16, 2026
% ============================================

% ============================================
% CSI403 Full Stack Development
% Common Preamble for All Presentations
% SPU Chonburi - Semester 2/2568
% ============================================

\documentclass[aspectratio=169,11pt]{beamer}

% ============================================
% PACKAGES
% ============================================
% Theme
\usepackage{../common/beamerthemeSPU}

% Graphics
\usepackage{graphicx}
\graphicspath{{./images/}{../common/images/}}

% TikZ for diagrams
\usepackage{tikz}
\usetikzlibrary{shapes,arrows,positioning,fit,calc,shadows,decorations.pathreplacing}

% Tables
\usepackage{booktabs}
\usepackage{tabularx}
\usepackage{multirow}
\usepackage{colortbl}

% Code listings
\usepackage{listings}
\usepackage{minted}
\usemintedstyle{friendly}

% Boxes
\usepackage{tcolorbox}
\tcbuselibrary{skins,breakable}

% Math
\usepackage{amsmath}
\usepackage{amssymb}

% Icons
\usepackage{fontawesome5}

% Hyperlinks
\usepackage{hyperref}
\hypersetup{
    colorlinks=true,
    linkcolor=spumaroon,
    urlcolor=codeblue,
    citecolor=spumaroon
}

% ============================================
% LISTINGS SETTINGS
% ============================================
\lstset{
    basicstyle=\ttfamily\small,
    backgroundcolor=\color{codebackground},
    keywordstyle=\color{codeblue}\bfseries,
    stringstyle=\color{codegreen},
    commentstyle=\color{codegray}\itshape,
    numberstyle=\tiny\color{codegray},
    numbers=left,
    numbersep=5pt,
    breaklines=true,
    breakatwhitespace=true,
    tabsize=4,
    showspaces=false,
    showstringspaces=false,
    frame=single,
    framerule=0.5pt,
    rulecolor=\color{spulightmaroon},
    xleftmargin=15pt,
    framexleftmargin=15pt,
    aboveskip=10pt,
    belowskip=10pt
}

% Python style
\lstdefinestyle{python}{
    language=Python,
    morekeywords={self,True,False,None,as,with,yield,async,await},
    morestring=[b]""",
    morestring=[b]'''
}

% HTML style
\lstdefinestyle{html}{
    language=HTML,
    morekeywords={DOCTYPE,html,head,body,div,span,class,id,href,src}
}

% SQL style
\lstdefinestyle{sql}{
    language=SQL,
    morekeywords={INT,VARCHAR,DECIMAL,DATETIME,PRIMARY,KEY,FOREIGN,REFERENCES,
                  CREATE,TABLE,INSERT,INTO,VALUES,SELECT,FROM,WHERE,JOIN,ON,
                  UPDATE,SET,DELETE,DROP,ALTER,ADD,INDEX,CONSTRAINT}
}

% Bash style
\lstdefinestyle{bash}{
    language=bash,
    morekeywords={docker,docker-compose,pip,python,pytest,git,curl,cd,ls,mkdir}
}

% ============================================
% TCOLORBOX STYLES
% ============================================
\newtcolorbox{codebox}[1][]{
    colback=codebackground,
    colframe=spumaroon,
    boxrule=1pt,
    arc=3pt,
    fonttitle=\bfseries,
    title=#1
}

\newtcolorbox{infobox}[1][]{
    colback=spulightgray,
    colframe=codeblue,
    boxrule=1pt,
    arc=3pt,
    fonttitle=\bfseries,
    title=#1
}

\newtcolorbox{warningbox}[1][]{
    colback=orange!10,
    colframe=codeorange,
    boxrule=1pt,
    arc=3pt,
    fonttitle=\bfseries,
    title=#1
}

\newtcolorbox{successbox}[1][]{
    colback=green!10,
    colframe=codegreen,
    boxrule=1pt,
    arc=3pt,
    fonttitle=\bfseries,
    title=#1
}

% ============================================
% CUSTOM COMMANDS
% ============================================
% File path
\newcommand{\filepath}[1]{\texttt{\color{codepurple}#1}}

% Terminal command
\newcommand{\cmd}[1]{\texttt{\color{codegreen}\$ #1}}

% URL display
\newcommand{\urlbox}[1]{\fcolorbox{codeblue}{codebackground}{\texttt{\color{codeblue}#1}}}

% Keyboard key
\newcommand{\key}[1]{\fbox{\small\texttt{#1}}}

% Tech logo placeholder
\newcommand{\techlogo}[2][1cm]{\includegraphics[height=#1]{tech-icons/#2}}

% Week indicator
\newcommand{\weekheader}[2]{%
    \begin{tikzpicture}[remember picture,overlay]
        \node[anchor=north east,xshift=-0.5cm,yshift=-0.3cm,
              fill=spugold,text=spumaroon,font=\small\bfseries,
              rounded corners=3pt,inner sep=5pt] 
              at (current page.north east) {Week #1 | #2};
    \end{tikzpicture}
}

% Progress bar
\newcommand{\progressbar}[2]{%
    \begin{tikzpicture}
        \fill[spulightgray] (0,0) rectangle (10,0.3);
        \fill[spumaroon] (0,0) rectangle (#1,0.3);
        \node[anchor=west] at (10.2,0.15) {\small #2};
    \end{tikzpicture}
}

% ============================================
% TITLE PAGE LOGO
% ============================================
\titlegraphic{%
    \includegraphics[height=1.5cm]{spu_logo.png}
}

% ============================================
% COURSE INFORMATION
% ============================================
\newcommand{\coursetitle}{CSI403: Full Stack Development}
\newcommand{\coursesemester}{Semester 2/2568}
\newcommand{\instructor}{Aj. Methas Khamjad}
\newcommand{\university}{Sripatum University Chonburi}

% Default author/institute
\author[\instructor]{\instructor}
\institute[\university]{%
    School of Information Technology\\
    \university
}

\endinput


\title[Week 2: Planning]{Week 2: Project Planning \& Software Requirements}
\subtitle{CSI403 Full Stack Development}
\date{January 14-16, 2026}

\begin{document}

% Title
\begin{frame}[plain]
    \titlepage
\end{frame}

% Outline
\begin{frame}{Today's Agenda}
    \tableofcontents
\end{frame}

% ============================================
\section{Project Management Fundamentals}
% ============================================

\begin{frame}{Why Project Management?}
    \weekheader{2}{Jan 14-16}
    
    \begin{columns}[T]
        \begin{column}{0.5\textwidth}
            \begin{block}{Without Planning}
                \begin{itemize}
                    \item Missed deadlines
                    \item Budget overruns
                    \item Scope creep
                    \item Team confusion
                    \item Failed projects
                \end{itemize}
            \end{block}
        \end{column}
        \begin{column}{0.5\textwidth}
            \begin{block}{With Planning}
                \begin{itemize}
                    \item Clear milestones
                    \item Resource allocation
                    \item Risk management
                    \item Team alignment
                    \item Successful delivery
                \end{itemize}
            \end{block}
        \end{column}
    \end{columns}
    
    \vspace{0.5cm}
    \begin{center}
        \textbf{``Failing to plan is planning to fail''}
    \end{center}
\end{frame}

\begin{frame}{Software Development Life Cycle (SDLC)}
    \begin{center}
        \begin{tikzpicture}[scale=0.8]
            \foreach \i/\phase/\color in {
                0/Planning/spumaroon,
                1/Analysis/codeblue,
                2/Design/codegreen,
                3/Development/codeorange,
                4/Testing/codepurple,
                5/Deployment/spugold
            } {
                \node[circle, draw, thick, fill=\color!30, minimum size=1.8cm, font=\small] 
                    (\phase) at (\i*60:3.5) {\phase};
            }
            \draw[->, thick] (Planning) -- (Analysis);
            \draw[->, thick] (Analysis) -- (Design);
            \draw[->, thick] (Design) -- (Development);
            \draw[->, thick] (Development) -- (Testing);
            \draw[->, thick] (Testing) -- (Deployment);
            \draw[->, thick, dashed] (Deployment) -- (Planning);
            
            \node[fill=spumaroon, text=white, rounded corners, inner sep=5pt] at (Planning) {\tiny We are here!};
        \end{tikzpicture}
    \end{center}
\end{frame}

\begin{frame}{Agile vs Waterfall}
    \begin{columns}[T]
        \begin{column}{0.48\textwidth}
            \begin{block}{Waterfall}
                \begin{itemize}
                    \item Sequential phases
                    \item Heavy documentation
                    \item Fixed requirements
                    \item Late testing
                    \item \textcolor{spumaroon}{Traditional approach}
                \end{itemize}
            \end{block}
        \end{column}
        \begin{column}{0.48\textwidth}
            \begin{block}{Agile}
                \begin{itemize}
                    \item Iterative cycles
                    \item Working software
                    \item Flexible requirements
                    \item Continuous testing
                    \item \textcolor{codegreen}{Modern approach}
                \end{itemize}
            \end{block}
        \end{column}
    \end{columns}
    
    \vspace{0.5cm}
    \begin{alertblock}{Our Approach}
        We'll use a \textbf{hybrid approach}: Agile sprints with essential documentation.
    \end{alertblock}
\end{frame}

% ============================================
\section{Software Requirements Specification (SRS)}
% ============================================

\begin{frame}{What is SRS?}
    \begin{block}{Software Requirements Specification}
        A comprehensive document that describes \textbf{what} the software will do, not \textbf{how} it will do it.
    \end{block}
    
    \vspace{0.5cm}
    
    \textbf{Key Components:}
    \begin{enumerate}
        \item \textbf{Introduction} - Purpose, scope, definitions
        \item \textbf{Overall Description} - Product perspective, user classes
        \item \textbf{Functional Requirements} - What the system does
        \item \textbf{Non-Functional Requirements} - Performance, security
        \item \textbf{Use Cases} - User interactions
        \item \textbf{User Stories} - Feature descriptions
    \end{enumerate}
\end{frame}

\begin{frame}{Functional vs Non-Functional Requirements}
    \begin{columns}[T]
        \begin{column}{0.48\textwidth}
            \begin{block}{Functional (What it does)}
                \begin{itemize}
                    \item User can register account
                    \item User can apply for loan
                    \item Admin can approve loans
                    \item System calculates interest
                    \item User can make payments
                \end{itemize}
            \end{block}
        \end{column}
        \begin{column}{0.48\textwidth}
            \begin{block}{Non-Functional (How well)}
                \begin{itemize}
                    \item Response time < 2 seconds
                    \item 99.9\% uptime
                    \item Support 1000 concurrent users
                    \item Data encrypted at rest
                    \item Mobile responsive UI
                \end{itemize}
            \end{block}
        \end{column}
    \end{columns}
\end{frame}

\begin{frame}{User Stories Format}
    \begin{block}{Template}
        \centering
        \Large
        As a \textcolor{codeblue}{\textbf{[role]}},\\
        I want \textcolor{codegreen}{\textbf{[feature]}},\\
        So that \textcolor{codeorange}{\textbf{[benefit]}}.
    \end{block}
    
    \vspace{0.5cm}
    
    \begin{exampleblock}{Examples for Loan System}
        \begin{itemize}
            \item As a \textbf{borrower}, I want to \textbf{apply for a loan online}, so that \textbf{I don't need to visit a branch}.
            \item As an \textbf{admin}, I want to \textbf{view all pending applications}, so that \textbf{I can process them efficiently}.
            \item As a \textbf{borrower}, I want to \textbf{see my payment history}, so that \textbf{I can track my loan progress}.
        \end{itemize}
    \end{exampleblock}
\end{frame}

\begin{frame}{Use Case Diagram}
    \begin{center}
        \begin{tikzpicture}[scale=0.75]
            % System boundary
            \draw[thick, rounded corners] (-1,-4) rectangle (7,4);
            \node[above] at (3,4) {\textbf{Loan Management System}};
            
            % Actors
            \node[circle, draw, thick, minimum size=0.8cm] (borrower) at (-3,2) {};
            \node[below] at (-3,1.2) {\small Borrower};
            \draw (-3,1.8) -- (-3,0.5);
            \draw (-3.3,1.3) -- (-3,1.5) -- (-2.7,1.3);
            \draw (-3,0.5) -- (-3.3,0) (-3,0.5) -- (-2.7,0);
            
            \node[circle, draw, thick, minimum size=0.8cm] (admin) at (-3,-2) {};
            \node[below] at (-3,-2.8) {\small Admin};
            \draw (-3,-2.2) -- (-3,-3.5);
            \draw (-3.3,-2.7) -- (-3,-2.5) -- (-2.7,-2.7);
            \draw (-3,-3.5) -- (-3.3,-4) (-3,-3.5) -- (-2.7,-4);
            
            % Use cases
            \node[ellipse, draw, thick, fill=codeblue!20, minimum width=2.5cm] (register) at (3,3) {\small Register};
            \node[ellipse, draw, thick, fill=codeblue!20, minimum width=2.5cm] (login) at (3,1.5) {\small Login};
            \node[ellipse, draw, thick, fill=codegreen!20, minimum width=2.5cm] (apply) at (3,0) {\small Apply Loan};
            \node[ellipse, draw, thick, fill=codegreen!20, minimum width=2.5cm] (view) at (3,-1.5) {\small View Loans};
            \node[ellipse, draw, thick, fill=codeorange!20, minimum width=2.5cm] (approve) at (3,-3) {\small Approve Loan};
            
            % Connections
            \draw[->] (-2,2) -- (register);
            \draw[->] (-2,2) -- (login);
            \draw[->] (-2,1.5) -- (apply);
            \draw[->] (-2,1) -- (view);
            \draw[->] (-2,-2) -- (login);
            \draw[->] (-2,-2.5) -- (view);
            \draw[->] (-2,-3) -- (approve);
        \end{tikzpicture}
    \end{center}
\end{frame}

% ============================================
\section{Notion for Project Management}
% ============================================

\begin{frame}{Why Notion?}
    \begin{columns}[T]
        \begin{column}{0.5\textwidth}
            \begin{block}{Features}
                \begin{itemize}
                    \item Free for students
                    \item Kanban boards
                    \item Document collaboration
                    \item Database views
                    \item Timeline/Calendar
                    \item Team workspace
                \end{itemize}
            \end{block}
        \end{column}
        \begin{column}{0.45\textwidth}
            \begin{block}{What We'll Track}
                \begin{itemize}
                    \item User Stories
                    \item Sprint Tasks
                    \item Bug Reports
                    \item Meeting Notes
                    \item Documentation
                \end{itemize}
            \end{block}
        \end{column}
    \end{columns}
    
    \vspace{0.5cm}
    \begin{infobox}[Getting Started]
        \begin{enumerate}
            \item Create account at \urlbox{notion.so}
            \item Create team workspace
            \item Use template provided by instructor
        \end{enumerate}
    \end{infobox}
\end{frame}

\begin{frame}{Kanban Board Structure}
    \begin{center}
        \begin{tikzpicture}[scale=0.8]
            % Columns
            \foreach \i/\title/\color in {0/Backlog/codegray, 3/To Do/codeblue, 6/In Progress/codeorange, 9/Testing/codepurple, 12/Done/codegreen} {
                \draw[thick, fill=\color!20] (\i,0) rectangle (\i+2.5,5);
                \node[font=\small\bfseries] at (\i+1.25,4.5) {\title};
            }
            
            % Sample cards
            \node[draw, fill=white, rounded corners, font=\tiny, text width=2cm, align=center] at (1.25,3.5) {User Registration\\US-001};
            \node[draw, fill=white, rounded corners, font=\tiny, text width=2cm, align=center] at (1.25,2.5) {Login System\\US-002};
            \node[draw, fill=white, rounded corners, font=\tiny, text width=2cm, align=center] at (4.25,3.5) {Loan Form\\US-003};
            \node[draw, fill=white, rounded corners, font=\tiny, text width=2cm, align=center] at (7.25,3.5) {Dashboard\\US-004};
            \node[draw, fill=white, rounded corners, font=\tiny, text width=2cm, align=center] at (10.25,3.5) {DB Setup\\US-005};
            \node[draw, fill=white, rounded corners, font=\tiny, text width=2cm, align=center] at (13.25,3.5) {Project Init\\US-000};
        \end{tikzpicture}
    \end{center}
    
    \begin{block}{Card Information}
        Each card should include: Title, Description, Assignee, Due Date, Priority, Labels
    \end{block}
\end{frame}

% ============================================
\section{Python Review: Functions \& Loops}
% ============================================

\begin{frame}{Python Functions}
    \begin{codebox}[Function Definition]
        \begin{lstlisting}[style=python]
def calculate_interest(principal, rate, years):
    """
    Calculate simple interest.
    
    Args:
        principal: Loan amount
        rate: Annual interest rate (%)
        years: Loan term in years
    
    Returns:
        Total interest amount
    """
    interest = principal * (rate / 100) * years
    return interest

# Usage
result = calculate_interest(100000, 7.5, 3)
print(f"Interest: {result:,.2f} THB")
        \end{lstlisting}
    \end{codebox}
\end{frame}

\begin{frame}{Python Loops}
    \begin{columns}[T]
        \begin{column}{0.48\textwidth}
            \begin{codebox}[For Loop]
                \begin{lstlisting}[style=python,basicstyle=\ttfamily\tiny]
# Iterate over list
loans = [50000, 100000, 150000]

for loan in loans:
    print(f"Loan: {loan:,}")

# With index
for i, loan in enumerate(loans):
    print(f"{i+1}. {loan:,}")

# Range
for i in range(1, 6):
    print(f"Month {i}")
                \end{lstlisting}
            \end{codebox}
        \end{column}
        \begin{column}{0.48\textwidth}
            \begin{codebox}[While Loop]
                \begin{lstlisting}[style=python,basicstyle=\ttfamily\tiny]
# Payment simulation
balance = 100000
month = 0
payment = 10000

while balance > 0:
    month += 1
    balance -= payment
    if balance < 0:
        balance = 0
    print(f"Month {month}: {balance:,}")

print(f"Paid off in {month} months!")
                \end{lstlisting}
            \end{codebox}
        \end{column}
    \end{columns}
\end{frame}

\begin{frame}{Python Lists \& Dictionaries}
    \begin{codebox}[Data Structures]
        \begin{lstlisting}[style=python]
# List - ordered collection
loan_amounts = [50000, 100000, 150000, 200000]
loan_amounts.append(250000)  # Add item
total = sum(loan_amounts)    # Sum all

# Dictionary - key-value pairs
borrower = {
    "name": "John Doe",
    "income": 50000,
    "loan_amount": 100000,
    "status": "pending"
}

# Access values
print(borrower["name"])
print(borrower.get("email", "N/A"))  # Default value

# List of dictionaries
loans = [
    {"id": 1, "amount": 50000, "status": "current"},
    {"id": 2, "amount": 100000, "status": "paid"}
]
        \end{lstlisting}
    \end{codebox}
\end{frame}

% ============================================
\section{Assignment: G2 Project Planning}
% ============================================

\begin{frame}{G2 Assignment: Project Planning}
    \begin{alertblock}{Due: Week 3 (Presentation)}
        Create comprehensive project planning documents for your Loan Management System.
    \end{alertblock}
    
    \textbf{Deliverables:}
    \begin{enumerate}
        \item \textbf{SRS Document} (5-10 pages)
            \begin{itemize}
                \item Introduction \& Scope
                \item 5-10 User Stories
                \item Use Case Diagram
                \item Functional Requirements
                \item Non-Functional Requirements
            \end{itemize}
        \item \textbf{Notion Board}
            \begin{itemize}
                \item Kanban board with all user stories
                \item Sprint planning (Week 1-15)
                \item Team member assignments
            \end{itemize}
        \item \textbf{Project Timeline}
            \begin{itemize}
                \item Gantt chart or timeline view
                \item Key milestones marked
            \end{itemize}
    \end{enumerate}
\end{frame}

\begin{frame}{Grading Rubric (5\%)}
    \begin{center}
        \small
        \begin{tabular}{|l|c|l|}
            \hline
            \textbf{Criteria} & \textbf{Points} & \textbf{Description} \\
            \hline
            SRS Completeness & 2.0 & All sections present and detailed \\
            \hline
            User Stories Quality & 1.0 & Clear, measurable, testable \\
            \hline
            Use Case Diagram & 0.5 & Correct UML notation \\
            \hline
            Notion Board Setup & 1.0 & Properly organized, tasks assigned \\
            \hline
            Presentation & 0.5 & Clear explanation, Q\&A \\
            \hline
            \textbf{Total} & \textbf{5.0} & \\
            \hline
        \end{tabular}
    \end{center}
    
    \vspace{0.5cm}
    \begin{infobox}[Submission]
        \begin{itemize}
            \item SRS: Upload to GitHub as \filepath{docs/SRS.md}
            \item Notion: Share workspace link with instructor
            \item Present: Week 3 class (10 min per team)
        \end{itemize}
    \end{infobox}
\end{frame}

% ============================================
\section{Summary}
% ============================================

\begin{frame}{Week 2 Summary}
    \begin{columns}[T]
        \begin{column}{0.48\textwidth}
            \begin{block}{What We Learned}
                \begin{itemize}
                    \item[$\checkmark$] SDLC phases
                    \item[$\checkmark$] SRS components
                    \item[$\checkmark$] User Stories format
                    \item[$\checkmark$] Use Case diagrams
                    \item[$\checkmark$] Notion for PM
                    \item[$\checkmark$] Python review
                \end{itemize}
            \end{block}
        \end{column}
        \begin{column}{0.48\textwidth}
            \begin{alertblock}{Action Items}
                \begin{itemize}
                    \item[$\square$] Create Notion workspace
                    \item[$\square$] Draft User Stories
                    \item[$\square$] Write SRS document
                    \item[$\square$] Setup Kanban board
                    \item[$\square$] Prepare presentation
                \end{itemize}
            \end{alertblock}
        \end{column}
    \end{columns}
\end{frame}

\begin{frame}{Next Week Preview}
    \begin{block}{Week 3: HTML/CSS \& Bootstrap}
        \begin{itemize}
            \item HTML5 Structure \& Semantics
            \item CSS3 Styling \& Layouts
            \item Bootstrap 5 Components
            \item Responsive Design
            \item \textbf{G2 Presentations}
        \end{itemize}
    \end{block}
\end{frame}

% Final Slide
\begin{frame}[plain]
    \begin{center}
        \vspace{2cm}
        {\Huge\bfseries\color{spumaroon} Questions?}
        \vspace{1cm}
        
        {\Large Thank you!}
        \vspace{1cm}
        
        \faEnvelope\ methas@spuchonburi.ac.th\\[5pt]
        \faGithub\ github.com/csi403-fullstack
        \vspace{1cm}
        
        \includegraphics[height=1cm]{spu_logo.png}
    \end{center}
\end{frame}

\end{document}
